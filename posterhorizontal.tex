
\documentclass[final]{beamer}


\newcommand{\compresslist}{
	\setlength{\itemsep}{.5pt}
	\setlength{\parskip}{1pt}
	\setlength{\parsep}{0pt}
}

\usepackage[scale=0.8,size=a1]{beamerposter} % Use the beamerposter package for laying out the poster

\usetheme{confposter} % Use the confposter theme supplied with this template
\usepackage{multicol}
\setbeamercolor{block title}{fg=dblue,bg=white} % Colors of the block titles
\setbeamercolor{block body}{fg=black,bg=white} % Colors of the body of blocks
\setbeamercolor{block alerted title}{fg=white,bg=dblue!70} % Colors of the highlighted block titles
\setbeamercolor{block alerted body}{fg=black,bg=dblue!10} % Colors of the body of highlighted blocks
% Many more colors are available for use in beamerthemeconfposter.sty

%-----------------------------------------------------------
% Define the column widths and overall poster size
% To set effective sepwid, onecolwid and twocolwid values, first choose how many columns you want and how much separation you want between columns
% In this template, the separation width chosen is 0.024 of the paper width and a 4-column layout
% onecolwid should therefore be (1-(# of columns+1)*sepwid)/# of columns e.g. (1-(4+1)*0.024)/4 = 0.22
% Set twocolwid to be (2*onecolwid)+sepwid = 0.464
% Set threecolwid to be (3*onecolwid)+2*sepwid = 0.708
%%%%% XAVI %%%%%%%%%%%%%
\newcommand{\memo}[2]{\textcolor{#1}{#2}}
\newcommand{\xavi}[1]{\memo{orange}{#1\\}}

\newlength{\sepwid}
\newlength{\onecolwid}
\newlength{\twocolwid}
\newlength{\threecolwid}
\setlength{\paperwidth}{33.1in} % A0 width: 46.8in
\setlength{\paperheight}{23.4in} % A0 height: 33.1in
\setlength{\sepwid}{0.0\paperwidth} % Separation width (white space) between columns
\setlength{\onecolwid}{0.22\paperwidth} % Width of one column
\setlength{\twocolwid}{0.464\paperwidth} % Width of two columns
\setlength{\threecolwid}{0.708\paperwidth} % Width of three columns
\setlength{\topmargin}{-0.5in} % Reduce the top margin size
%-----------------------------------------------------------

\usepackage{graphicx}  % Required for including images

\usepackage{booktabs} % Top and bottom rules for tables
\usepackage{standalone} % to load standalone file (itkz picture for ex)
\usepackage{array} %finest gestion of tabular
\usepackage{algorithm,algorithmicx,algpseudocode}
\usepackage{setspace}

%----------------------------------------------------------------------------------------
%	TITLE SECTION 
%----------------------------------------------------------------------------------------

%\title{Exploring the dynamic of cultural changes: A model to understand the amphorae production patterns in the Roman Empire} % Poster title
%\title{New perspective on the study of variations in Amphorae production during the Roman Empire} % Poster title
\title{Cultural changes of baetican olive oil amphorae from an evolutionary perspective}


\author{Maria Coto-Sarmiento$^{1}$, Simon Carrignon$^{1,2}$ and Xavier Rubio-Campillo$^{3}$} % Author(s)

\institute{$^1$Barcelona Supercomputing Center -- $^2$Universitat Pompeu Fabra -- $^3$University of Edinburgh} % Institution(s)

%----------------------------------------------------------------------------------------

\begin{document}

\addtobeamertemplate{block end}{}{\vspace*{2ex}} % White space under blocks
\addtobeamertemplate{block alerted end}{}{\vspace*{2ex}} % White space under highlighted (alert) blocks

\setlength{\belowcaptionskip}{2ex} % White space under figures
\setlength\belowdisplayshortskip{2ex} % White space under equations

\begin{frame}[t] % The whole poster is enclosed in one beamer frame

\vspace{-1cm}

\begin{columns}[t] % The whole poster consists of three major columns, the second of which is split into two columns twice - the [t] option aligns each column's content to the top

\begin{column}{\sepwid}\end{column} % Empty spacer column

\begin{column}{\onecolwid} % The first column


\begin{block}{Introduction}

\justify

Archaeological indicators of variability in material culture can explain how cultural processes evolve over time and space~\cite{lipo}. In particular, our study aims to analyse the changes in the amphorae production patterns during the Roman Empire. This study explores the question of transmission on the learning techniques processes. Specifically, we are interested in understanding if pottery-making techniques were transmitted through vertical or horizontal social learning. If vertical transmission predominates in this process then amphorae made in nearby workshops may share more similar traits than amphorae from farthest workshops~\cite{bjo}. We also analysed the correlation between spatial distance and morphometric variation by exploring test. In this work we have explored the social learning processes associated with amphorae production through a combination of empirical analysis and theoretical exploration.   


\end{block}

\vspace{-0.5cm}

\begin{block}{Materials}

\justify
We have analyzed 470 amphorae (fig.\ref{fig:betica}~(b)) from 5 different workshops located in \emph{Baetica} (fig.\ref{fig:betica}~(a)). This province supplied a massive quantity of olive oil to the rest of the Empire from the Ist to the IIIrd centuries, and for this reason a large-scale infrastructure of amphorae production was developed here. The same amphoric type (Dressel 20) was produced in several workshops located along the course of the Guadalquivir river. A sample of 90 amphorae was chosen for each of the four analysed workshops. Eight different measures were taken for each amphorae, most of them focused on the rim as an indicator of more variability. 


%The workshops were selected from different sites of \emph{Baetica} province in order to know if morphometric distance was correlated with spatial distance\xavi{you can remove this sentence, you already said that they are from Baetica, and that the aim of the study is to see if variation is correlated with distance. A better phrasing:The same amphoric type (Dressel 20) was produced in several workshops located along the course of the Guadalquivir river}
%\xavi{A sample is the entire set of amphorae, not each amphora, so you can't say "90 sample".}

\begin{figure}
\begin{tabular}{cc}


\includegraphics[width=0.7\linewidth]{images/fig1.png} &
\includegraphics[width=0.2\linewidth]{images/amphorae.png} \\
(a) & (b)
\end{tabular}

\singlespace
\caption{a) More than 80 pottery workshops were distributed along the Guadalquivir river and its tributary the Genil. Red points correspond to the workshops analyzed. b) Dressel 20 amphora}
\label{fig:betica}
\end{figure}


 \end{block}
\end{column} % End of the first column

%BEGIN THE SECOND COLUMN-------------------------------------------------
\begin{column}{\twocolwid}


\begin{block}{Empirical Analysis}

\begin{columns}[t,totalwidth=\twocolwid]
\begin{column}{\onecolwid} %first subcolumn left


{\textbf{Methods}} 
\justify

Principal Component Analysis (PCA) was used to capture most of the variance of the 8 measurements into 2 variables. 


\end{column}

\begin{column}{\sepwid}\end{column} % Empty spacer column

\begin{column}{\onecolwid} %first subcolumn right

{\textbf{Results}}\\
\justify
The pattern observed in the firsts 2 Principal Components suggests that amphorae from closer workshops tend to be more similar (see Figure~\ref{fig:pca}). In this case, the four closest workshops show variation on PC1 (i.e. Bel\'en, Delicias, Villaseca and Malpica) while Parlamento displays a distinctive pattern than the rest of workshops on PC2 values.


\end{column}
 % End of the second column
\end{columns}

\begin{columns}[t,totalwidth=\twocolwid]


\begin{column}{\twocolwid} %first subcolumn left
\begin{figure}
\includegraphics[width=0.65\linewidth]{images/fig2.pdf}
\singlespace
\caption{First and Second Principal Components for amphorae measured from the 5 analysed workshops}
\label{fig:pca}
\end{figure}
\end{column}
\end{columns}
\end{block}
\vspace{-1cm}
\begin{block}{Theoretical Exploration}

\begin{columns}[t,totalwidth=\twocolwid]

\begin{column}{1.025\onecolwid} %first subcolumn left
%on the left
{\textbf{Model}}
\justify
We developed a model based on classical random drift~\cite{bentley2004randomdriftandculturechange} mimicking our original dataset. In this model, we define $5$ workshops all sharing the same production techniques $T^{0}$. 
Each workshop produces amphorae and changes their production techniques by: modifying their own techniques (Vertical Transmission, VT) or by copying one from another workshop (Horizontal Transmission, HT). 

We used the more significant measurements (External Diameter \& Protruding Rim) taken from the original dataset as the amphora's representation in the model.

%The production of the amphora is represented by two measurements (External Diameter \& Protruding Rim) defined using the original dataset.
\vspace{1cm}


To test the influence of HT and VT on the patterns observed we designed three setups:\\
\begin{enumerate}
    \item  \hspace{5mm}{\footnotesize\textbf{VT}}: probability of random copy between workshop is set to~$0$. Changes only come from vertical transmission.\\
    \item \hspace{5mm}{\footnotesize\textbf{VT+HT(d)}}: probability of random copy \emph{proportional to distance} between workshops. \\
    \item  \hspace{5mm}{\footnotesize\textbf{VT+HT}}: probability of random copy \emph{equal} between all workshops. \\
\end{enumerate}
%Amphorae are describe by on trait and we look how this trait varies between each workshop at the end of the simulation.
\end{column}

\begin{column}{1.025\onecolwid} %first subcolumn right
{\textbf{Result}}\\
\justify
\vspace{-1cm}
    \begin{figure}[h!]
	\begin{tabular}{cc}
	    \centering
	    \small   External Diameter & \small Protruding Rim\\
	    \includegraphics[width=0.5\linewidth]{images/ED_densities.pdf}&
	    \includegraphics[width=0.5\linewidth]{images/PR_densities.pdf}\\
	\end{tabular}
\singlespace
\vspace{-.8cm}
\caption{Inter-workshop variation of the mean of the trait after $10\,000$ timestep for both model (100 simulations per model). The red line correspond to the variation measured on the real data}
	\label{fig:resmod}
    \end{figure}
Fig.~\ref{fig:resmod} shows that when vertical transmission is only acting ({\footnotesize\textbf{VT}}), the variation is higher than the observed variation: the techniques diverge more and more due to random drift. On the other hand, when horizontal transmission is not correlated to spatial distance ({\footnotesize\textbf{VT+HT}}), variation is too low: all workshops tend to use the same techniques and produce similar amphorae. Finally it is when the horizontal transmission is biased toward the distance between workshops ({\footnotesize\textbf{VT+HT(d)}}), that the variation gets closer to the one observed in the dataset.
\end{column}
\end{columns}

\end{block}
\end{column}
% End of the second column

%BEGIN LAST COLUMN----------------------------------------------------

%\begin{column}{\sepwid}\end{column} % Empty spacer column

\begin{column}{\onecolwid} % The third column

\begin{block}{Discussion}
\justify

Empirical studies have identified variation on the learning processes among pottery workshops. We observe that this variability is correlated with spatial distance: the analysed morphometric traits show that the similarity between amphorae decrease with the spatial distance between the workshops they were produced.

The combination between empirical data with theoretical model suggests that the similarity on the morphological traits are produced by a continuous contact between closest workshops, as in the case of \textbf{VT+HT(d)}. By contrast, when we use the model where \textbf{HT} is not correlated to spatial distance the variation is lower than the empirical data. However, this variability in traits is higher than the real data when \textbf{VT} is the only one mechanism of transmission without contact between workshops. 

This could be interpreted by the fact that pottery techniques were learned from master to disciple at the beginning and these potters exchanged the learning techniques with the nearest workshops. 




%vertical transmission would be the most probable main cultural mechanism of learning at the beginning.This could be interpreted by the fact that pottery techniques were learned from master to disciple. After simulating, the variation of the morphological selected traits among workshops was higher in vertical transmission than the real data. A low contact between workshops (VT) stimulated a high variation in morphological traits. By contrast, a continuous contact between workshops  provided a low variability of traits.


%This could be interpreted by the fact that pottery techniques were learned from master to disciple and that disciples remained in the workshop where they were trained.
%The low different morphological traits among workshops do not seem proper of a low contact between potters from others workshops.


\end{block}

\begin{block}{References}
\small

\begin{thebibliography}{50}


\bibitem[1]{lipo}\textsc{Eerkens, J. \& Lipo, C.P (2005)}
\textit{Cultural transmission, copying errors, and the generation of variation in material culture and the archaeological records}, Journal of Anthropological Archaeology, 24, 316-334.

\bibitem[2]{bjo}\textsc{Bj\"{o}rklund, M., Bergek, S., Ranta, E., \& Kaitala, V. (2010)}
\textit{The effect of local population dynamics on patterns of isolation by distance}, Ecological Informatics, 5(3), 167-172.

\bibitem[3]{bentley2004randomdriftandculturechange}\textsc{Bentley, R., Hahn, M., Shennan, S. (2004)}
\textit{Random drift and culture change}, Proceedings of the Royal Society of London. Series B: Biological Sciences, 271(1547), 1443--1450.


\end{thebibliography}
%	\scriptsize
%	\renewcommand{\refname}{\vspace{-0.5em}}
%	//\bibliographystyle{abbrv}
%	\bibliography{biblio}

\end{block}

%----------------------------------------------------------------------------------------
%	ACKNOWLEDGEMENTS
%----------------------------------------------------------------------------------------

\setbeamercolor{block title}{fg=dblue,bg=white} % Change the block title color

\begin{block}{Acknowledgements}

\small{\rmfamily{The Funding for this work was provided by the ERC Advanced Grant EPNet (340828).}}

\end{block}

%----------------------------------------------------------------------------------------
%	CONTACT INFORMATION
%----------------------------------------------------------------------------------------

\setbeamercolor{block alerted title}{fg=white,bg=dblue!70} % Change the alert block title colors
\setbeamercolor{block alerted body}{fg=black,bg=white} % Change the alert block body colors


\begin{center}
\begin{tabular}{ccc}
\includegraphics[width=0.51\linewidth]{images/epnet.png} & \hfill & \includegraphics[width=0.45\linewidth]{images/erc.png}
\end{tabular}
\end{center}

%----------------------------------------------------------------------------------------

\end{column} % End of the third column

\end{columns} % End of all the columns in the poster

\end{frame} % End of the enclosing frame

\end{document}
